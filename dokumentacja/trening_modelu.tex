\section{Trening modelu}

Proces trenowania modelu realizowany jest przez niewielką aplikację (\textit{Kotlin, OpenNLP NER API}),której zadaniem jest dostarczenie danych próbki, ustawienie parametrów uczenia, wystartowanie procesu trenowania, oraz zapisanie wygenerowanego binarnego pliku modelu. Najistotniejszy fragment aplikacji przedstawiony jest na listingu. 
\small
\begin{lstlisting}
private fun trenujModel(aZbiorRegul:Path): TokenNameFinderModel {
	// reading training data
	var inputFactory: InputStreamFactory?
	try {
		inputFactory =
		 MarkableFileInputStreamFactory(aZbiorRegul.toFile())
		} catch (e: FileNotFoundException) {
			throw IllegalArgumentException(e)
		}

	var sampleStream: ObjectStream<*>?

	sampleStream = NameSampleDataStream(
	PlainTextByLineStream(inputFactory, StandardCharsets.UTF_8))

	// setting the parameters for train ing
	val params = TrainingParameters()
	params.put(TrainingParameters.ITERATIONS_PARAM, 70)
	params.put(TrainingParameters.CUTOFF_PARAM, 1)

	// training the model using TokenNameFinderModel class
	var nameFinderModel: TokenNameFinderModel?
	try {
		nameFinderModel = NameFinderME.train("en"
		, null
		, sampleStream
		,params
		, TokenNameFinderFactory.create(null
					, null
					, mutableMapOf<String, Any>()
					, BioCodec()))

	return nameFinderModel

	} catch (e: IOException) {
		throw IllegalArgumentException(e)
	}
}
\end{lstlisting}
	
\normalsize
Raport podsumowujący pojedynczą sesję treningową wykonaną w oparciu o przygotowaną wcześniej próbkę wygląda następująco:

\small
\begin{lstlisting}

> Task :app-modul-silnik-regul:wytrenujModel
=====encje_reguly_probka_uczaca.reg
Indexing events with TwoPass using cutoff of 1

Computing event counts...  done. 259140 events
Indexing...  done.
Collecting events... Done indexing in 6,59 s.
Incorporating indexed data for training...  
done.
Number of Event Tokens: 259140
Number of Outcomes: 13
Number of Predicates: 654
Computing model parameters...
Performing 70 iterations.
1:  . (259100/259140) 0.9998456432816238
2:  . (259140/259140) 1.0
3:  . (259140/259140) 1.0
4:  . (259140/259140) 1.0
5:  . (259140/259140) 1.0
Stopping: change in training set accuracy less than 1.0E-5
Stats: (259140/259140) 1.0
...done.
Compressed 654 parameters to 322
79 outcome patterns
Trained model saved to file in location=>src/main/resources/modelnlp/encje_reguly_model.bin
\end{lstlisting}
\normalsize
\paragraph{}\noindent
Trening modelu jest krokiem kończącym etap przygotowawczy. Zapisany plik binarny jest gotowy do użycia go w aplikacji. W następnej części postaram się opisać jak można posłużyć się nim do rozwiązania konkretnego problemu.