\documentclass[12pt, a4paper, twoside, titlepage]{article}

\usepackage[T1]{fontenc}

\usepackage{polski}
\usepackage[utf8]{inputenc}
\usepackage[polish]{babel}
\usepackage[below]{placeins}

\usepackage{amsmath} 
\usepackage{mathtools}
\usepackage[margin=0.5in]{geometry}
\usepackage{hyperref}
\usepackage{listings}

\newcommand{\textunderbrace}[2]{%
	\ensuremath{\underbrace{\small{\text{#1}}}_{\text{#2}}}%
}
\newcommand{\textoverbrace}[2]{%
	\ensuremath{\overbrace{\text{#1}}^{\text{#2}}}%
}

\newcommand{\stext}[1]{%		
		\begin{lstlisting}
			zawartość...
		\end{lstlisting}
		#1
	}%


% font size could be 10pt (default), 11pt or 12 pt
% paper size coulde be letterpaper (default), legalpaper, executivepaper,
% a4paper, a5paper or b5paper
% side coulde be oneside (default) or twoside 
% columns coulde be onecolumn (default) or twocolumn
% graphics coulde be final (default) or draft 
%
% titlepage coulde be notitlepage (default) or titlepage which 
% makes an extra page for title 
% 
% paper alignment coulde be portrait (default) or landscape 
%
% equations coulde be 
%   default number of the equation on the rigth and equation centered 
%   leqno number on the left and equation centered 
%   fleqn number on the rigth and  equation on the left side
%	
\title{Realizacja prostego silnika reguł walidacyjnych przy użyciu technik NLP}
\author{Mariusz Wójcik
}

\date{\today} 
% \date{\today} date coulde be today 
% \date{25.12.00} or be a certain date
% \date{ } or there is no date 
\begin{document}
	% Hint: \title{what ever}, \author{who care} and \date{when ever} could stand 
	% before or after the \begin{document} command 
	% BUT the \maketitle command MUST come AFTER the \begin{document} command! 
	\maketitle
	
	
	%\begin{abstract}
	%	Poniższy dokument jest opisem realizacji zagadnienia technik przetwarzania języka 
%	Do realizacji tych zadań wykorzystana została biblioteka \hyperref['https://opennlp.apache.org/']{''OpenNLP''} %dostarczająca narzędzi do przetwarzania zdań zapisanych w języku naturalnym.
 				
%	\end{abstract}
	
	%\tableofcontents % create a table of contens 
	
	
	
	
	\section{Początki}
Jakiś czas temu miałem okazję obejrzeć film, który zrobił na mnie ogromne wrażenie. \textit{,,Arrival - Nowy Początek''} w reżyserii Denisa Villeneuve’a.
Porusza on problematykę szerokopojętej, wielopłaszczyznowej komunikacji (a czasem skutków jej braku) . W niesamowicie sugestywny i obrazowy sposób pokazuje mechanizm kształtowania się podstaw wspólnego języka i nawiązywania kontaktu.
Proces stopniowego budowania uwspólnionych modeli pojęciowych prowadzący do porozumiewania się tym samym językiem wydał mi się tak logiczny i uporządkowany, że sprawiał wrażenie niemal algorytmicznego...

\paragraph{}
Wtedy jeszcze wydawało mi się, że porozumiewanie się językiem naturalnym jest domeną przynależną wyłącznie człowiekowi.
Miałem poczucie, że prace nad komputerowym przetwarzaniem języka naturalnego mają wymiar wyłącznie akademicki.
Bardzo szybko przekonałem się jak bardzo się myliłem.O kazało się, że dynamiczny rozwój algorytmów sztucznej inteligencji i
przetwarzania maszynowego dotknął również tej dziedziny. Gdzieś na styku matematyki, informatyki i lingwistyki wykształciła się
dziedzina, która funkcjonuje jako\textit{ NLP (ang.  natural language processing )}.
	
\paragraph{}
Techniki NLP koncentrują się na analizie, przekształcaniu i generowaniu języka naturalnego.
Dzięki nim komputery nabywają umiejętności nie tylko analizy tekstu, ale również nauki i wyciągania wniosków.
Dają one możliwość analizy nie tylko składni zdań, ale również doszukiwania się ich znaczeń i ukrytych pomiędzy słowami intencji.
Czasami uświadamiam sobie że to wszystko razem brzmi jak czysta fantastyka. Bo jak niby sens, znaczenie i intencje można przeliczyć na liczby i twardo
zakotwiczyć w dziedzinie algebry liniowej ?
\paragraph{}
Tajemnicy tej uchyla jedna z najciekawszych książek, jaką miałem przyjemność ostatnio czytać, mianowicie \textit{,,Natural Language Processing in Action''}.
Jest to bardzo przystępnie napisany przewodnik, dzięki któremu łatwiej oswoić się z podstawowymi prawami rządzącymi światem NLP.
Pozycja nie traktuje o rzeczach najłatwiejszych, a mimo to czyta się ją z dużą przyjemnością.
\paragraph{}
Z teorią często jest tak, że w którymś momencie chciałoby się ją zobaczyć w praktyce. Z tej potrzeby zrodził się pomysł na aplikację, którą możnaby
zrealizować przy użyciu technik i algorytmów NLP.
\newline
Przyszedł mi do głowy generator kodu aplikacji, który byłby w stanie przekształcić tekst napisany językiem zbliżonym do naturalnego bezpośrednio do kodu wykonywalnego.  Oczywiście zakładam że tego typu rozwiązanie miałoby zastosowanie do jakiegoś ściśle określonego aspektu działającej aplikacji, np. walidacji dokumentu, czy sprawdzania reguł poprawności modelu dziedziny.  
\newline
I tak właśnie powstał mój miniprojekt, którego celem jest zobaczenie o co tak naprawdę chodzi z tym NLP . :) . Zapraszam do zapoznania się z założeniami i otrzymanymi wynikami.
\paragraph{}
Mam świadomość, że jeśli chodzi o NLP, jestem na początku drogi. Nie mogę powiedzieć nawet tego, że udało mi się zrobić jeden krok, ale wiem jedno...
zapowiada się naprawdę pasjonująca podróż...
	\section{Realizacja}
Do realizacji moich założeń wybrałem napisaną w Javia bibliotekę \textit{Apache OpenNLP}. Dostarcza ona narzędzi realizujących wiele aspektów przetwarzania języka naturalnego.
W moim projekcie skupię się technice nazywanej \textit{Named Entity Recognition (NER)}. 
\\
Polega ona na rozpoznawaniu w tekście określonych bytów nazwanych. Najczęściej są to imiona, nazwiska, nazwy własne itp.  W moim przypadku chciałbym stworzyć własny model, który zostanie przyuczony do rozpoznawania poszczególnych elementów konstrukcji reguły walidacyjnej (takich jak słowa kluczowe rozpoczynające i kończące bloki, operatory, akcje i ich parametry). 
\\
Żeby to osiągnąć konieczne jest przygotowanie odpowiednio opisanej próbki uczącej, a następnie wykorzystanie jej do treningu modelu.

	\section{Abstrakcyjny model reguły}
Kluczowym elementem całego rozwiązania jest model, w oparciu o który narzędzia dostępne w bibliotece OpenNLP będą dokonywały analizy reguł. By przygotować taki model konieczne jest dostarczenie dobrej jakości próbki uczącej, która weźmie udział w procesie jego trenowania. 
\\
Przygotowanie próbki rozpocznę od opracowania schematu reguły. Oczekuję, że system prawidłowo rozpozna poszczególne składowe każdej reguły, która będzie z nim zgodna. 


Na początek wypiszę sobie kilka przykładowych reguł walidacyjnych.
\\ \\
\fbox{\begin{minipage}{40em}
		
		\begin{enumerate}
		\item Jeśli wiek\_pacjenta jest większy od 18 wtedy zgłoś błąd ,,Pacjent jest osobą dorosłą.'', w przeciwnym wypadku
		wyświetl komunikat ,,Pacjent został zakwalifikowany do leczenia pediatrycznego.''.
		\item Jeśli data\_kwalifikacji jest jest mniejsza od '01-01-2019' wtedy zgłoś wyjątek ,,Data sprzed roku 2019.'', w przeciwnym wypadku sprawdź regułę RS-001. 
		\item Gdy saldo\_rachunku jest większe od 100 oraz saldo\_rachunku jest mniejsze niż 1000 wtedy wyświetl komunikat ,,Saldo rachunku jest prawidłowe.'', w przeciwnym razie zgłoś błąd ,,Nieprawidłowe saldo rachunku''.
		\item Jeśli data\_teraz jest niewiększa niż data\_ważności wyświetl komunikat ,,Wniosek jest aktualny.'' w przeciwnym wypadku zgłaszaj błąd ,,Wniosek utracił ważność''.
	\end{enumerate}
	
\end{minipage}}
\\ \\

%\textunderbrace{Ala ma kota}{warunek}
%\textunderbrace{Text \textoverbrace{text}{aaa} text text}{bbb}

%\bigskip

%\emph{\textunderbrace{Text \textoverbrace{text}{aaa} text text}{bbb}}

%tesr\newline\\

Przyjmuję uproszczenie, że każda rozpoznawana reguła składała się będzie z trzech wyróżnialnych bloków:
\\ \\
\fbox{\begin{minipage}{40em}
\[
\underbrace{WARUNKI}  \underbrace{AKCJA\_TAK}  \underbrace{(AKCJA\_NIE)?}
\]
%S\label{fig:test}test
\end{minipage}}
\\ \\

Poszczególne bloki oddzielone będą od siebie słowami kluczowymi oznaczającymi rozpoczęcie i zakończenie bloku. \\

W celu ich wyróżnienia wprowadzam następujące oznaczenia:
\begin{enumerate}
	\item SK\_SW - Start sekcji warunku
	\item SK\_KW - Koniec sekcji warunku
	\item SK\_SAN - Start sekcji akcji wykonywanej przy niespełnionym warunku
\end{enumerate}
Schemat reguły przyjmuje następującą postać:
\\ \\
\fbox{\begin{minipage}{40em}
		\[
		\underbrace{SK\_SW} \underbrace{WARUNEK} \underbrace{SK\_KW} \underbrace{AKCJA\_TAK}  (\underbrace{SK\_SAN}\underbrace{AKCJA\_NIE)?}
		\]
		
\end{minipage}}
\\ \\

Powyższy schemat można odnieść do przykładowej reguły:
\\ \\
\fbox{\begin{minipage}{40em}
			\[
		\textunderbrace{Jeśli}{SK\_SW} 			 
		\]
		\[
		\textunderbrace{data jest jest mniejsza od '01-01-2019' lub data jest większa niż '01-06-2019'}{WARUNEK} 		 
		\]
		\[
		\textunderbrace{wtedy}{SK\_KW} 		
		\]
		\[ 		
		\textunderbrace{zgłoś wyjątek ,,Data spoza dopuszczonego przedziału.''}{AKCJA\_TAK}		
		\]
			\[		
		\textunderbrace{w przeciwnym wypadku }{SK\_SAN} 	
		\]
		\[			
		\textunderbrace{sprawdź regułę RS-001.}{AKCJA\_NIE} 
		\]
		
\end{minipage}}
\\ \\

\paragraph{}
Ponieważ kluczowe jest właściwe rozpoznanie sekcji warunku, chciałbym wyłączyć go przed nawias i przez chwilę skupić się wyłącznie na nim. 

%In Figure \ref{fig:test} we see a beautiful duck!
	\section{Przygotowanie próbki uczącej}
Mechanizmy NER wchodzące w skład \textit{OpenNLP} pozwalają na stworzenie własnego modelu i przyuczenie go do rozpoznawania specyficznych bytów domenowych. Próbka ucząca jest dosyć obszernym zbiorem przykładów (dokumentacja\textit{ OpenNLP} mówi o minimum 15 tyś. zdań), w których w specjalny sposób otagowane zostały kluczowe frazy. 
\\ \\
\fbox{\begin{minipage}{40em}
		\small{			
		<START:SK\_SW> jeśli <END> <START:OP\_L> xxx <END> <START:OPR\_REL> jest większy niż <END> <START:OP\_P> xxx <END> <START:SK\_KW> wtedy <END> <START:AKCJA> zgłoś błąd <END> <START:AKCJA\_PARAMETR> xxx <END> .
	}			
\end{minipage}}

\paragraph{}
Byty, które docelowo mają być rozpoznawane przez model należy umieścić pomiędzy tagami \\ \textit{<START:NAZWA\_BYTU>} i \textit{<END>}. Dodatkowo każda reguła danych uczących zbudowana jest według schematu omawianego wcześniej abstrakcyjnego modelu reguły. Zgodne z nim są również nazwy encji.
W przypadku tych części reguły, które są zmienne i specyficzne dla każdej instancji (takie jak komentarze, nazwy operandów, wszelkie parametry) użyłem frazy \textit{xxx}, która oznacza że będzie tu coś, o czym na tym etapie nie możemy nic powiedzieć (znamy tylko pozycję tego tokena względem innych encji ). 
\paragraph{}
Wygenerowanie próbki uczącej okazało się zagadnieniem samym w sobie. Do tego celu napisałem aplikację pythonową, która w danych wejściowych otrzymuje przewidywane przykładowe wartości encji,a następnie otagowuje je, tworzy ich  iloczyny kartezjańskie i ostatecznie konstruuje z nich zdanie zgodne z założonym schematem.

Poniżej znajdują się wartości poszczególnych encji użyte to wygenerowania danych uczących.

\[
\hspace*{-22em}
SK\_SW=
\begin{Bmatrix*}[l]
\text{jeśli}\\
\text{gdy}\\
\text{jeżeli}
\end{Bmatrix*}
\]

\[ 
\hspace*{-20em}
OPR\_REL=
\begin{Bmatrix*}[l]
	\text{nie}
\end{Bmatrix*}
*
\begin{Bmatrix*}[l]
	\text{jest równy}\\
	\text{jest równa}\\
	\text{jest równe}\\
	\text{jest większy}\\
	\text{jest mniejszy}\\
	\text{jest różny}\\
	\text{jest większa}\\
	\text{jest mniejsza}\\
	\text{jest różna}\\
	\text{jest większe}\\
	\text{jest mniejsze}\\
	\text{jest różne}
\end{Bmatrix*}
\begin{Bmatrix*}[l]
\text{niż}\\
\text{od}
\end{Bmatrix*}
\]

\[	
\hspace*{-22em}
SK\_KW=
	\begin{Bmatrix*}[l]
	\text{wtedy}\\
	\text{to}
\end{Bmatrix*}
\]	


\[
\hspace*{-15em}
	SK\_SAN=
	\begin{Bmatrix*}[l]
	\text{w przeciwnym wypadku}
	\end{Bmatrix*}
\]

\[
\hspace*{-16em}
\begin{array}{l}
AKCJA=
	\begin{Bmatrix*}[l]
	\text{zgłoś błąd}\\
	\text{zgłoś błąd walidacji}\\
	\text{raportuj błąd}\\
	\text{wyświetl komunikat}\\
	\text{sprawdź regułę}\\
	\text{sprawdzaj regułę}
	\end{Bmatrix*}\\

	\end{array}
\]

\paragraph{}

Przykładowe, bardziej złożone reguły utworzone opisaną wcześniej techniką:

\begin{enumerate}
	\item \small\textit{ <START:SK\_SW> jeśli <END> <START:OP\_L> xxx <END> <START:OPR\_REL> jest większy niż <END> <START:OP\_P> xxx <END> <START:SK\_KW> wtedy <END> <START:AKCJA> zgłoś błąd <END> <START:AKCJA\_PARAMETR> xxx <END>  <START:SK\_SAN> w przeciwnym wypadku <END> <START:AKCJA> sprawdzaj regułę <END> <START:AKCJA\_PARAMETR> xxx <END> .}
	\item \small\textit{ <START:SK\_SW> jeśli <END> <START:OP\_L> xxx <END> <START:OPR\_REL> nie jest mniejsza od <END> <START:OP\_P> xxx <END> <START:OPR\_LOG> oraz <END> <START:OP\_L> xxx <END> <START:OPR\_REL> nie jest równy <END> <START:OP\_P> xxx <END> <START:SK\_KW> wtedy <END> <START:AKCJA> zgłoś błąd <END> <START:AKCJA\_PARAMETR> xxx <END>  <START:SK\_SAN> w przeciwnym wypadku <END> <START:AKCJA> zgłoś błąd <END> \\ <START:AKCJA\_PARAMETR> xxx <END> .}
	
	\item \small \textit{<START:SK\_SW> jeśli <END> <START:OP\_L> xxx <END> <START:OPR\_REL> jest większy niż <END> <START:OP\_P> xxx <END> <START:OPR\_LOG> lub <END> <START:OP\_L> xxx <END> <START:OPR\_REL> jest różny od <END> <START:OP\_P> xxx <END> <START:SK\_KW> wtedy <END> <START:AKCJA> zgłoś błąd <END> <START:AKCJA\_PARAMETR> xxx <END>  <START:SK\_SAN> w przeciwnym wypadku <END> <START:AKCJA> zgłoś błąd walidacji <END> <START:AKCJA\_PARAMETR> xxx <END> .}
\end{enumerate}

Liczność próbki użytej do treningu modelu ustawiłem na poziomie ok 20 000 rekordów.
	\section{Trening modelu}

Proces trenowania modelu realizowany jest przez niewielką aplikację (\textit{Kotlin, OpenNLP NER API}),której zadaniem jest dostarczenie danych próbki, ustawienie parametrów uczenia, wystartowanie procesu trenowania, oraz zapisanie wygenerowanego binarnego pliku modelu. Najistotniejszy fragment aplikacji przedstawiony jest na listingu. 
\small
\begin{lstlisting}
private fun trenujModel(aZbiorRegul:Path): TokenNameFinderModel {
	// reading training data
	var inputFactory: InputStreamFactory?
	try {
		inputFactory =
		 MarkableFileInputStreamFactory(aZbiorRegul.toFile())
		} catch (e: FileNotFoundException) {
			throw IllegalArgumentException(e)
		}

	var sampleStream: ObjectStream<*>?

	sampleStream = NameSampleDataStream(
	PlainTextByLineStream(inputFactory, StandardCharsets.UTF_8))

	// setting the parameters for train ing
	val params = TrainingParameters()
	params.put(TrainingParameters.ITERATIONS_PARAM, 70)
	params.put(TrainingParameters.CUTOFF_PARAM, 1)

	// training the model using TokenNameFinderModel class
	var nameFinderModel: TokenNameFinderModel?
	try {
		nameFinderModel = NameFinderME.train("en"
		, null
		, sampleStream
		,params
		, TokenNameFinderFactory.create(null
					, null
					, mutableMapOf<String, Any>()
					, BioCodec()))

	return nameFinderModel

	} catch (e: IOException) {
		throw IllegalArgumentException(e)
	}
}

\end{lstlisting}
	
\normalsize
Raport podsumowujący pojedynczą sesję treningową wykonaną w oparciu o przygotowaną wcześniej próbkę wygląda następująco:

\small
\begin{lstlisting}

> Task :app-modul-silnik-regul:wytrenujModel
=====encje_reguly_probka_uczaca.reg
Indexing events with TwoPass using cutoff of 1

Computing event counts...  done. 259140 events
Indexing...  done.
Collecting events... Done indexing in 6,59 s.
Incorporating indexed data for training...  
done.
Number of Event Tokens: 259140
Number of Outcomes: 13
Number of Predicates: 654
Computing model parameters...
Performing 70 iterations.
1:  . (259100/259140) 0.9998456432816238
2:  . (259140/259140) 1.0
3:  . (259140/259140) 1.0
4:  . (259140/259140) 1.0
5:  . (259140/259140) 1.0
Stopping: change in training set accuracy less than 1.0E-5
Stats: (259140/259140) 1.0
...done.
Compressed 654 parameters to 322
79 outcome patterns
Trained model saved to file in location=>src/main/resources/modelnlp/encje_reguly_model.bin

\end{lstlisting}
\label{raport_trening}
\normalsize
\paragraph{}\noindent
Trening modelu jest krokiem kończącym etap przygotowawczy. Zapisany plik binarny jest gotowy do użycia go w aplikacji. W następnej części postaram się opisać jak można posłużyć się nim do rozwiązania konkretnego problemu.
	\section{Pytania bez odpowiedzi}
Ze względu na fakt, że mój miniprojekt ma charakter czysto akademicki, to uzyskane wyniki oceniam jako bardzo ciekawe i zadowalające. Mam jednak świadomość (a może tylko mi się wydaje że mam \dots), że zaprezentowana przeze mnie metoda rozwiązania problemu transkrypcji języka naturalnego na kod wykonywalny nie jest doskonała, a wiele pojawiających się zagadnień wymaga gruntownego przestudiowania. Poniżej omówię kilka zagadnień problemowych, które na tę chwilę pozostają bez rozwiązania. 
\subsection{Ocena jakości danych uczących}
Pierwsza rzecz, która sprawia mi trudność to ocena jakości dostarczonej przeze mnie próbki uczącej. Jak ocenić, czy dane nie są zbyt schematyczne, a wytrenowana sieć zachowuje swoje zdolności generalizacyjne. Może dobór przykładów, lub lepsza ich konstrukcja wpłynęłaby na poprawę osiągniętych wyników. Nie znam sposobu na zinterpretowanie dosyć lakonicznego raportu z procesu uczenia \ref{raport_trening}. Zastanowienie budzi również niska wartość prawdopodobieństwa rozpoznawalności poszczególnych encji. Dlaczego bardzo rzadko jest ona większa niż 0.2 ?
\subsection{Rozszerzalność próbki danych uczących }
Kolejna rzecz, która wymaga poprawy to sposób rozszerzania zakresu rozpoznawalnych przez algorytmy schematów. Co należałoby zrobić by sieć zaczęła rozpoznawać inny typ warunku np. frazy typu ,,czy pole xxx zostało zdefiniowane'', albo jak sprawić by sieć potrafiła rozpoznać n - grup warunkowych połączonych operatorem logicznym (w tej chwili radzi sobie z dwoma warunkami połączonymi jednym operatorem). Czy dostarczenie kolejnej liczby przykładów obejmujących taki rodzaj warunku na pewno rozwiąże problem ? Jeśli tak to jak ocenić optymalną dla próbki liczbę nowodostarczonych przykładów. Czy liczba przykładów w nowym schemacie nie powinna zachować jakiejś proporcji względem liczby próbek w starym schemacie ? Tu obawiałbym się sytuacji, że zbyt duża liczba próbek w nowym schemacie mógłaby spowodować że stanie się on dominujący, co z kolei doprowadziłoby do pogorszenia dotychczasowych osiągnięć. Pozatym takie dołączanie kolejnych przypadków szybko doprowadzi do zbyt dużego rozdrobnienia przykładów i lawinowy rozrost liczności próbki.
\subsection{Optymalny dobór parametrów uczenia}
Sam proces treningu posiada możliwości parametryzacji. Ja skorzystałem z tych najbardziej podstawowych. Może bardziej świadomy ich dobór doprowadziłby do otrzymania lepszych wyników.
\subsection{Odporność na błędy i zaburzenia}
Tutaj największym problemem jest utrzymanie zgodności z założonym schematem, oraz ocena czy proces rozpoznania przebiegł poprawnie. Nie znam na to innego sposobu niż ocena ekspercka, oczyma człowieka. Jak ustrzec się przed tym, że osoba wpisująca regułę nie wzbogaci jej treści we frazy, które są całkowicie nierozpoznawalne. Proces rozpoznawania encji i tak się zakończy powodzeniem, ale tokeny mogą zostać rozpoznane nieprawidłowo. Jak zdiagnozować taką sytuację ?
\subsection{Optymalny dobór narzędzi i technologii}
W tym punkcie należałoby się zastanowić, czy wybór biblioteki \textit{OpenNLP, techniki NER}, oraz algorytmu jest optymalny do rozwiązania tego typu zadań. Co prawda w świecie Java możliwości wyboru bibliotek są bardziej ograniczone niż w środowisku \textit{Python}, to jednak należałoby zrobić pod tym kątem rozpoznanie chociażby narzędzi \textit{Stanford NLP}.
	\input{ocena_wynikow.tex}
%	\input{wzor.tex}






	
\end{document}