\documentclass[a4paper,landscape]{article}
\usepackage[utf8]{inputenc} % kodowanie ISO Latin2 (Pod Win cp1250)
%\usepackage[MeX]{polski}
\usepackage[polish]{babel}


\usepackage{amsmath} 
\usepackage{mathtools}
\usepackage[margin=0.5in]{geometry}
%opening
\title{}
\author{}

\begin{document}




%\[
%z = \overbrace{
%	\underbrace{x}_\text{real} +
%	\underbrace{iy}_\text{imaginary}
% }^\text{complex number}
%\]

SK\_WARUNKI\_START SEKWENCJA\_WARUNKOW SK\_WARUNKI\_STOP SEKCJA\_AKCJA\_TAK  SEKCJA\_AKCJA\_NIE
\[POCZATEK\_REGULY
 \underbrace{\hspace*{10pt} WARUNEK \hspace*{10pt}  (OPERATOR\_LOGICZNY \hspace*{10pt} WARUNEK)* \hspace*{10pt}}_\text{Sekwencja warunków} 
\ KONIEC\_REGULY 
\]


Gdzie 

\[
WARUNEK=	\underbrace{LEWY\_OPERAND\_WARUNKU \hspace*{20pt} OPERATOR\_WARUNKU \hspace*{20pt} PRAWY\_OPERAND\_WARUNKU }_\text{WARUNEK}
\]

%\[
%\left( x-1 \right) \underbrace{ \left( ? \right) }_{ \mathclap{\text{what is this?}} }
%\]


%\section{title}
%POCZATEK\_REGULY\ WARUNEK (OPERATOR_LOGICZNY WARUNEK) * KONIEC_REGULY AKCJA (PARAMETR_AKCJI) %W_PRZECIWNYM_RAZIE AKCJA (PARAMETR_AKCJI).

%Gdzie

%WARUNEK=LEWOSTRONNY_OPERAND_WARUNKU OPERATOR_POROWNANIA PRAWOSTRONNY_OPERAND_WARUNKU


\end{document}
