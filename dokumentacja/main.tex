\documentclass[12pt, a4paper, twoside, titlepage]{article}

\usepackage[T1]{fontenc}

\usepackage{polski}
\usepackage[utf8]{inputenc}
\usepackage[polish]{babel}
\usepackage[below]{placeins}

\usepackage{amsmath} 
\usepackage{mathtools}
\usepackage[margin=0.5in]{geometry}
\usepackage{hyperref}

\newcommand{\textunderbrace}[2]{%
	\ensuremath{\underbrace{\small{\text{#1}}}_{\text{#2}}}%
}
\newcommand{\textoverbrace}[2]{%
	\ensuremath{\overbrace{\text{#1}}^{\text{#2}}}%
}

% font size could be 10pt (default), 11pt or 12 pt
% paper size coulde be letterpaper (default), legalpaper, executivepaper,
% a4paper, a5paper or b5paper
% side coulde be oneside (default) or twoside 
% columns coulde be onecolumn (default) or twocolumn
% graphics coulde be final (default) or draft 
%
% titlepage coulde be notitlepage (default) or titlepage which 
% makes an extra page for title 
% 
% paper alignment coulde be portrait (default) or landscape 
%
% equations coulde be 
%   default number of the equation on the rigth and equation centered 
%   leqno number on the left and equation centered 
%   fleqn number on the rigth and  equation on the left side
%	
\title{Realizacja prostego silnika reguł walidacyjnych przy użyciu technik NLP}
\author{Mariusz Wójcik
}

\date{\today} 
% \date{\today} date coulde be today 
% \date{25.12.00} or be a certain date
% \date{ } or there is no date 
\begin{document}
	% Hint: \title{what ever}, \author{who care} and \date{when ever} could stand 
	% before or after the \begin{document} command 
	% BUT the \maketitle command MUST come AFTER the \begin{document} command! 
	\maketitle
	
	
	%\begin{abstract}
	%	Poniższy dokument jest opisem realizacji zagadnienia technik przetwarzania języka 
%	Do realizacji tych zadań wykorzystana została biblioteka \hyperref['https://opennlp.apache.org/']{''OpenNLP''} %dostarczająca narzędzi do przetwarzania zdań zapisanych w języku naturalnym.
 				
%	\end{abstract}
	
	%\tableofcontents % create a table of contens 
	
	
	
	
	\section{Początki}
Jakiś czas temu miałem okazję obejrzeć film, który zrobił na mnie ogromne wrażenie. ,,Arrival - Nowy Początek'' w reżyserii Denisa Villeneuve’a - to obraz niezwykły.
Porusza on problematykę szerokopojętej, wielopłaszczyznowej komunikacji (a czasem skutków jej braku) . W niesamowicie sugestywny i obrazowy sposób pokazuje mechanizm kształtowania się podstaw wspólnego języka i nawiązywania kontaktu.
Proces stopniowego budowania uwspólnionych modeli pojęciowych prowadzący do porozumiewania się tym samym językiem wydał mi się tak logiczny i uporządkowany, że sprawiał wrażenie niemal algorytmicznego...
Pamiętam swoją myśl, że skoro możliwe jest tak precyzyjne określenie
reguł stojących u podstaw nawiązania skutecznej komunikacji, to droga do zbudowania inteligentnych maszyn porozumiewających się z nami ,,ludzkim''
językiem wydaje się już bardzo krótka.

\paragraph{}
Do niedawna wydawało mi się, że porozumiewanie się językiem naturalnym jest domeną przynależną wyłącznie człowiekowi.
Miałem poczucie, że prace nad komputerowym przetwarzaniem języka naturalnego mają wymiar wyłącznie akademicki.
Okazało się jednak, że dynamiczny rozwój algorytmów sztucznej inteligencji i
przetwarzania maszynowego dotknął również tej dziedziny. Gdzieś na styku matematyki, informatyki i lingwistyki wykształciła się
dziedzina, która funkcjonuje jako NLP (ang.  natural language processing ).
	
\paragraph{}
Techniki NLP koncentrują się na analizie, przekształcaniu i generowaniu języka naturalnego.
Dzięki nim komputery nabywają umiejętności nie tylko analizy tekstu, ale również nauki i wyciągania wniosków.
Dają one możliwość analizy nie tylko składni zdań, ale również doszukiwania się ich znaczeń i ukrytych pomiędzy słowami intencji.
Czasami uświadamiam sobie że to wszystko razem brzmi jak czysta fantastyka. Bo jak niby sens, znaczenie i intencje można przeliczyć na liczby i twardo
zakotwiczyć w dziedzinie algebry liniowej ?
\paragraph{}
Tajemnicy tej uchyla jedna z najciekawszych książek, jaką miałem przyjemność ostatnio czytać, mianowicie ,,Natural Language Processing in Action''.
Jest to bardzo przystępnie napisany przewodnik, dzięki któremu łatwiej oswoić się z podstawowymi prawami rządzącymi światem NLP.
Pozycja nie traktuje o rzeczach najłatwiejszych, a mimo to czyta się ją z dużą przyjemnością.
\paragraph{}
Z teorią często jest tak, że w którymś momencie chciałoby się ją zobaczyć w praktyce. Z tej potrzeby zrodził się pomysł na aplikację, którą możnaby
zrealizować przy użyciu technik i algorytmów NLP.
\newline
Przyszedł mi do głowy generator kodu aplikacji, który byłby w stanie przekształcić tekst napisany językiem zbliżonym do naturalnego bezpośrednio do kodu wykonywalnego.  Oczywiście zakładam że tego typu rozwiązanie miałoby zastosowanie do jakiegoś ściśle określonego aspektu działającej aplikacji, np. walidacji dokumentu, czy sprawdzania reguł poprawności modelu dziedziny.  
\newline
I tak właśnie powstał mój miniprojekt, którego celem jest zobaczenie o co tak naprawdę chodzi z tym NLP . :) . Zapraszam do zapoznania się z założeniami i otrzymanymi wynikami.
\paragraph{}
Mam świadomość, że jeśli chodzi o NLP, jestem na początku drogi. Nie mogę powiedzieć nawet tego, że udało mi się zrobić jeden krok, ale wiem jedno...
podróż zapowiada się naprawdę imponująco...
	\section{Założenia i sposób realizacji}
	\section{Abstrakcyjny model reguły}

Przygotowanie próbki rozpocznę od opracowania schematu reguły. 


Na początek wypiszę sobie kilka przykładowych reguł walidacyjnych.
\\ \\
\fbox{\begin{minipage}{40em}
		
		\begin{enumerate}
		\item Jeśli wiek\_pacjenta jest większy od 18 wtedy zgłoś błąd ,,Pacjent jest osobą dorosłą.'', w przeciwnym wypadku
		wyświetl komunikat ,,Pacjent został zakwalifikowany do leczenia pediatrycznego.''.
		\item Jeśli data\_kwalifikacji jest jest mniejsza od '01-01-2019' wtedy zgłoś wyjątek ,,Data sprzed roku 2019.'', w przeciwnym wypadku sprawdź regułę RS-001. 
		\item Gdy saldo\_rachunku jest większe od 100 oraz saldo\_rachunku jest mniejsze niż 1000 wtedy wyświetl komunikat ,,Saldo rachunku jest prawidłowe.'', w przeciwnym razie zgłoś błąd ,,Nieprawidłowe saldo rachunku''.
		\item Jeśli data\_teraz jest niewiększa niż data\_ważności wyświetl komunikat ,,Wniosek jest aktualny.'' w przeciwnym wypadku zgłaszaj błąd ,,Wniosek utracił ważność''.
	\end{enumerate}
	
\end{minipage}}
\\ \\

%\textunderbrace{Ala ma kota}{warunek}
%\textunderbrace{Text \textoverbrace{text}{aaa} text text}{bbb}

%\bigskip

%\emph{\textunderbrace{Text \textoverbrace{text}{aaa} text text}{bbb}}

%tesr\newline\\

Przyjmuję uproszczenie, że każda rozpoznawana reguła składała się będzie z trzech wyróżnialnych bloków:
\\ \\
\fbox{\begin{minipage}{40em}
\[
\underbrace{WARUNKI}  \underbrace{AKCJA\_TAK}  \underbrace{(AKCJA\_NIE)?}
\]
%S\label{fig:test}test
\end{minipage}}
\\ \\

Poszczególne bloki oddzielone będą od siebie słowami kluczowymi oznaczającymi rozpoczęcie i zakończenie bloku. \\

W celu ich wyróżnienia wprowadzam następujące oznaczenia:
\begin{enumerate}
	\item SK\_SW - Start sekcji warunku
	\item SK\_KW - Koniec sekcji warunku
	\item SK\_SAN - Start sekcji akcji wykonywanej przy niespełnionym warunku
\end{enumerate}
Schemat reguły przyjmuje następującą postać:
\\ \\
\fbox{\begin{minipage}{40em}
		\[
		\textunderbrace{SK\_SW}{} \textunderbrace{WARUNKI}{} \textunderbrace{SK\_KW}{} \textunderbrace{AKCJA\_TAK}{}  (\textunderbrace{SK\_SAN}{}\textunderbrace{AKCJA\_NIE)?}{}
		\]
		
\end{minipage}}
\\ \\

Rzut oka na przykład:
\\ \\
\fbox{\begin{minipage}{40em}
			\[
		\textunderbrace{Jeśli}{SK\_SW} 			 
		\]
		\[
		\textunderbrace{data jest mniejsza od '01-01-2019' lub data jest większa niż '01-06-2019'}{WARUNEK} 		 
		\]
		\[
		\textunderbrace{wtedy}{SK\_KW} 		
		\]
		\[ 		
		\textunderbrace{zgłoś wyjątek ,,Data spoza dopuszczonego przedziału.''}{AKCJA\_TAK}		
		\]
			\[		
		\textunderbrace{w przeciwnym wypadku }{SK\_SAN} 	
		\]
		\[			
		\textunderbrace{sprawdź regułę RS-001.}{AKCJA\_NIE} 
		\]
		
\end{minipage}}
\\ \\

\paragraph{}
Ponieważ kluczowe jest właściwe rozpoznanie sekcji warunku, chciałbym wyłączyć go przed nawias i przez chwilę skupić się wyłącznie na nim. 
\\ \\
\fbox{\begin{minipage}{40em}
		
		\[
		\textunderbrace{data jest  mniejsza od '01-01-2019' lub data jest większa niż '01-06-2019'}{WARUNEK} 		 
		\]	
		
\end{minipage}}
\\ \\

Na początek trzeba zauważyć, że powyższa sekcja składa się z dwóch niezależnych wyrażeń warunkowych połączonych operatorem logicznym \textit{lub} . Każdy z pojedynczych warunków składa się z kolei z operatora relacyjnego \textit{(jest mniejsza, jest większa)}, oraz z dwóch operandów \textit{(lewego i prawego)}.
\\
Wprowadzam więc następujące oznaczenia:
\begin{enumerate}
	\item OP\_L - Operand lewy
	\item OPR\_REL - Operator relacyjny
	\item OP\_P - Operand prawy
	\item OPR\_LOG - Operator logiczny
\end{enumerate}

Po podstawieniu, przykładowy warunek można  
\\ \\
\fbox{\begin{minipage}{40em}
		
		\[
		\textunderbrace{\textunderbrace{data}{OP\_L} \textunderbrace{jest  mniejsza od}{OPR\_REL} \textunderbrace{'01-01-2019'}{OP\_P} \textunderbrace{lub}{OPR\_LOG} \textunderbrace{data}{OP\_L} \textunderbrace{jest większa niż}{OPR\_REL} \textunderbrace{'01-06-2019'}{OP\_P}}{WARUNEK} 		 
		\]	
		
\end{minipage}}
\\ \\
Zatem zapis symboliczny sekcji warunkowej będzie wyglądał następująco:
\\ \\
\fbox{\begin{minipage}{40em}
		
		\[
		\textunderbrace{OP\_L OPR\_REL OP\_P (OPR\_LOG OP\_L OPR\_REL OP\_P)?}{WARUNEK} 		 
		\]	
		
\end{minipage}}
\\ \\

Po dokonaniu podstawienia w sekcji \textit{WARUNKI} abstrakcyjny model reguły przyjmie następującą postać:
\\ \\
\fbox{\begin{minipage}{40em}
		\[
		\underbrace{SK\_SW}
		\]
		\[	
			\textunderbrace{OP\_L OPR\_REL OP\_P (OPR\_LOG OP\_L OPR\_REL OP\_P)?}{WARUNEK}		
		\]
		\[
			\underbrace{SK\_KW}
		\]
		\[
		\underbrace{AKCJA\_TAK}
		\]
		\[
		(\underbrace{SK\_SAN}\underbrace{AKCJA\_NIE)?}
		\]
		
\end{minipage}}
\\ \\
Jeśli chodzi o akcje, to sprawa wydaje się prostsza, bo każda z nich składa się z części mówiącej o tym co ma być zrobione (AKCJA), oraz z jakim parametrem ma być wykonane (AKCJA\_PARAMETR). Ostatecznie więc, po wykonaniu podstawienia nasz model przyjmie następującą, ostateczną formę:
\\ \\
\fbox{\begin{minipage}{40em}
	\[
		\underbrace{SK\_SW}
		\]
		\[	
		\textunderbrace{OP\_L OPR\_REL OP\_P (OPR\_LOG OP\_L OPR\_REL OP\_P)?}{WARUNEK}		
		\]
		\[
		\underbrace{SK\_KW}
		\]
		\[
		\textunderbrace{AKCJA AKCJA\_PARAMETR}{AKCJA\_TAK}
		\]
		\[
		(\underbrace{SK\_SAN}\textunderbrace{AKCJA AKCJA\_PARAMETR}{AKCJA\_NIE})?
		\]
		
		\begin{enumerate}
			\item SK\_SW - Start sekcji warunku
			\item SK\_KW - Koniec sekcji warunku
			\item SK\_SAN - Start sekcji akcji wykonywanej przy niespełnionym warunku
				\item OP\_L - Operand lewy
			\item OPR\_REL - Operator relacyjny
			\item OP\_P - Operand prawy
			\item OPR\_LOG - Operator logiczny
		\end{enumerate}
		
\end{minipage}}
\\ \\
I jeszcze ostatnie spojrzenie na przykład:
\\ \\
\fbox{\begin{minipage}{40em}
		\[
		\textunderbrace{Jeśli}{SK\_SW} 			 
		\]
		\[
	\textunderbrace{\textunderbrace{data}{OP\_L} \textunderbrace{jest  mniejsza od}{OPR\_REL} \textunderbrace{'01-01-2019'}{OP\_P} \textunderbrace{lub}{OPR\_LOG} \textunderbrace{data}{OP\_L} \textunderbrace{jest większa niż}{OPR\_REL} \textunderbrace{'01-06-2019'}{OP\_P}}{WARUNEK} 		 
	\]	
		\[
		\textunderbrace{wtedy}{SK\_KW} 		
		\]
		\[ 		
		\textunderbrace{\textunderbrace{zgłoś wyjątek}{AKCJA} \textunderbrace{,,Data spoza dopuszczonego przedziału.''}{AKCJA\_PARAMETR}}{AKCJA\_TAK}		
		\]
		\[		
		\textunderbrace{w przeciwnym wypadku }{SK\_SAN} 	
		\]
		\[			
		\textunderbrace{\textunderbrace{sprawdź regułę}{AKCJA} \textunderbrace{ RS-001.}{AKCJA\_PARAMETR}}{AKCJA\_NIE} 
		\]
		
\end{minipage}}
\\ \\

Tak określony model może być podstawą do wygenerowania odpowiednio oznakowanej próbki uczącej.

%	
\section{opis}

	
	\subsection{more introduction}
	Go more in detail \ldots
	
	\subsubsection{even more introduction}
	come to the point \ldots
	
	\paragraph{Paragraphs}
	A paragraph is small but 
	
	\subparagraph{Subparagraphs}
	subparagraphs are smaller! 
	
	\paragraph{Outline}
	First we start with a little example of the article class, which is an 
	important documentclass. But there would be other documentclasses like 
	book \ref{book}, report \ref{report} and letter \ref{letter} which are 
	described in Section \ref{documentclasses}. Finally, Section 
	\ref{conclusions} gives the conclusions.
	
	
	
	\section{Documentclasses} \label{documentclasses}
	
	\begin{itemize}
		\item article
		\item book 
		\item report 
		\item letter 
	\end{itemize}
	
	
	\begin{enumerate}
		\item article
		\item book 
		\item report 
		\item letter 
	\end{enumerate}
	
	\begin{description}
		\item[article\label{article}]{Article is \ldots}
		\item[book\label{book}]{The book class \ldots}
		\item[report\label{report}]{Report gives you \ldots}
		\item[letter\label{letter}]{If you want to write a letter.}
	\end{description}
	
	\section{tabular}
	No paper without a tabular!
	
	\begin{tabular}{|l|c|r|p{2cm}|}
		\hline
		first column & second column & third column & fourth column \\
		\hline 
		l stand for left & c for center & r for right & and p for predefined size \\
		\hline 
	\end{tabular} 
	
	
	\section{some math}
	Math in text is called in line math just put \$ character around 
	the math think. Like $ a^2 + b^2 = c^2 $. It looks better if you use 
	this 
	\[a^2 + b^2 = c^2\]
	
	\section{Conclusions}\label{conclusions}
	There is no longer \LaTeX{} example which was written by \cite{doe}.
	
	\begin{thebibliography}{9}
		\bibitem[Doe]{doe} \emph{First and last \LaTeX{} example.},
		John Doe 50 B.C. 
	\end{thebibliography}






	
\end{document}