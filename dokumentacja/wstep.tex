\section{Początki}
Jakiś czas temu miałem okazję obejrzeć film, który zrobił na mnie ogromne wrażenie. \textit{,,Arrival - Nowy Początek''} w reżyserii Denisa Villeneuve’a.
Porusza on problematykę szerokopojętej, wielopłaszczyznowej komunikacji (a czasem skutków jej braku) . W niesamowicie sugestywny i obrazowy sposób pokazuje mechanizm kształtowania się podstaw wspólnego języka i nawiązywania kontaktu.
Proces stopniowego budowania uwspólnionych modeli pojęciowych prowadzący do porozumiewania się tym samym językiem wydał mi się tak logiczny i uporządkowany, że sprawiał wrażenie niemal algorytmicznego...

\paragraph{}
Wtedy jeszcze wydawało mi się, że porozumiewanie się językiem naturalnym jest domeną przynależną wyłącznie człowiekowi.
Miałem poczucie, że prace nad komputerowym przetwarzaniem języka naturalnego mają wymiar wyłącznie akademicki.
Bardzo szybko przekonałem się jak bardzo się myliłem.O kazało się, że dynamiczny rozwój algorytmów sztucznej inteligencji i
przetwarzania maszynowego dotknął również tej dziedziny. Gdzieś na styku matematyki, informatyki i lingwistyki wykształciła się
dziedzina, która funkcjonuje jako\textit{ NLP (ang.  natural language processing )}.
	
\paragraph{}
Techniki NLP koncentrują się na analizie, przekształcaniu i generowaniu języka naturalnego.
Dzięki nim komputery nabywają umiejętności nie tylko analizy tekstu, ale również nauki i wyciągania wniosków.
Dają one możliwość analizy nie tylko składni zdań, ale również doszukiwania się ich znaczeń i ukrytych pomiędzy słowami intencji.
Czasami uświadamiam sobie że to wszystko razem brzmi jak czysta fantastyka. Bo jak niby sens, znaczenie i intencje można przeliczyć na liczby i twardo
zakotwiczyć w dziedzinie algebry liniowej ?
\paragraph{}
Tajemnicy tej uchyla jedna z najciekawszych książek, jaką miałem przyjemność ostatnio czytać, mianowicie \textit{,,Natural Language Processing in Action''}.
Jest to bardzo przystępnie napisany przewodnik, dzięki któremu łatwiej oswoić się z podstawowymi prawami rządzącymi światem NLP.
Pozycja nie traktuje o rzeczach najłatwiejszych, a mimo to czyta się ją z dużą przyjemnością.
\paragraph{}
Z teorią często jest tak, że w którymś momencie chciałoby się ją zobaczyć w praktyce. Z tej potrzeby zrodził się pomysł na aplikację, którą możnaby
zrealizować przy użyciu technik i algorytmów NLP.
\newline
Przyszedł mi do głowy generator kodu aplikacji, który byłby w stanie przekształcić tekst napisany językiem zbliżonym do naturalnego bezpośrednio do kodu wykonywalnego.  Oczywiście zakładam że tego typu rozwiązanie miałoby zastosowanie do jakiegoś ściśle określonego aspektu działającej aplikacji, np. walidacji dokumentu, czy sprawdzania reguł poprawności modelu dziedziny.  
\newline
I tak właśnie powstał mój miniprojekt, którego celem jest zobaczenie o co tak naprawdę chodzi z tym NLP . :) . Zapraszam do zapoznania się z założeniami i otrzymanymi wynikami.
\paragraph{}
Mam świadomość, że jeśli chodzi o NLP, jestem na początku drogi. Nie mogę powiedzieć nawet tego, że udało mi się zrobić jeden krok, ale wiem jedno...
zapowiada się naprawdę pasjonująca podróż...