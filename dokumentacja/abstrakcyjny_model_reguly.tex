\section{Abstrakcyjny model reguły}
Kluczowym elementem całego rozwiązania jest model, w oparciu o który narzędzia dostępne w bibliotece OpenNLP będą dokonywały analizy reguł. By przygotować taki model konieczne jest dostarczenie dobrej jakości próbki uczącej, która weźmie udział w procesie jego trenowania. 

Przygotowanie próbki rozpocznę od opracowania schematu reguły. Oczekuję, że system prawidłowo rozpozna poszczególne składowe każdej reguły, która będzie z nim zgodna. 

\paragraph{}
Na początek wypiszę sobie kilka przykładowych reguł walidacyjnych.

\fbox{\begin{minipage}{40em}
		\begin{enumerate}
		\item Jeśli wiek\_pacjenta jest większy od 18 wtedy zgłoś błąd ,,Pacjent jest osobą dorosłą.'', w przeciwnym wypadku
		wyświetl komunikat ,,Pacjent został zakwalifikowany do leczenia pediatrycznego.''.
		\item Jeśli data\_kwalifikacji jest jest mniejsza od '01-01-2019' wtedy zgłoś wyjątek ,,Data sprzed roku 2019.'', w przeciwnym wypadku sprawdź regułę RS-001. 
		\item Gdy saldo\_rachunku jest większe od 100 oraz saldo\_rachunku jest mniejsze niż 1000 wtedy wyświetl komunikat ,,Saldo rachunku jest prawidłowe.'', w przeciwnym razie zgłoś błąd ,,Nieprawidłowe saldo rachunku''.
		\item Jeśli data\_teraz jest niewiększa niż data\_ważności wyświetl komunikat ,,Wniosek jest aktualny.'' w przeciwnym wypadku zgłaszaj błąd ,,Wniosek utracił ważność''.
	\end{enumerate}
	
\end{minipage}}

\paragraph{}

%\textunderbrace{Ala ma kota}{warunek}
%\textunderbrace{Text \textoverbrace{text}{aaa} text text}{bbb}

%\bigskip

%\emph{\textunderbrace{Text \textoverbrace{text}{aaa} text text}{bbb}}

%tesr\newline\\

Przyjmuję uproszczenie, że dla każdej rozpoznawanej reguły możliwe będzie wydzielenie trzech bloków:
\\ \\
\fbox{\begin{minipage}{40em}
\[
\underbrace{WARUNKI}  \underbrace{AKCJA\_TAK}  \underbrace{(AKCJA\_NIE)?}
\]
\end{minipage}}
\\ \\
Poszczególne bloki oddzielone będą od siebie słowami kluczowymi oznaczającymi rozpoczęcie i zakończenie bloku. \\

Wprowadzam następujące oznaczenia:
\begin{enumerate}
	\item SW - Start sekcji warunku
	\item KW - Koniec sekcji warunku
	\item SAN - Start sekcji akcji wykonywanej przy niespełnionym warunku
\end{enumerate}
Schemat reguły przyjmuje następującą postać:
\\ \\
\fbox{\begin{minipage}{40em}
		\[
		\text{SW} \underbrace{WARUNKI} \text{KW} \underbrace{AKCJA\_TAK}  \underbrace{(\text{SAN }AKCJA\_NIE)?}
		\]
\end{minipage}}
\\ \\

W odniesieniu do przykładu wyglądałoby to tak:
\\ \\
\fbox{\begin{minipage}{40em}
		\[
		\textunderbrace{Jeśli}{SW} 	
		\textunderbrace{data\_kwalifikacji jest jest mniejsza od '01-01-2019'}{WARUNEK} 		 
		\]
		\[
		\textunderbrace{wtedy}{KW} 		
		\textunderbrace{zgłoś wyjątek ,,Data sprzed roku 2019.''}{AKCJA\_TAK}		
		\]
		\[		
		\textunderbrace{w przeciwnym wypadku }{SAN} 	
		\textunderbrace{sprawdź regułę RS-001.}{AKCJA\_NIE} 
		\]
\end{minipage}}
\\ \\
