\section{Abstrakcyjny model reguły}

Przygotowanie próbki rozpocznę od opracowania schematu reguły. 


Na początek wypiszę sobie kilka przykładowych reguł walidacyjnych.
\\ \\
\fbox{\begin{minipage}{40em}
		
		\begin{enumerate}
		\item Jeśli wiek\_pacjenta jest większy od 18 wtedy zgłoś błąd ,,Pacjent jest osobą dorosłą.'', w przeciwnym wypadku
		wyświetl komunikat ,,Pacjent został zakwalifikowany do leczenia pediatrycznego.''.
		\item Jeśli data\_kwalifikacji jest jest mniejsza od '01-01-2019' wtedy zgłoś wyjątek ,,Data sprzed roku 2019.'', w przeciwnym wypadku sprawdź regułę RS-001. 
		\item Gdy saldo\_rachunku jest większe od 100 oraz saldo\_rachunku jest mniejsze niż 1000 wtedy wyświetl komunikat ,,Saldo rachunku jest prawidłowe.'', w przeciwnym razie zgłoś błąd ,,Nieprawidłowe saldo rachunku''.
		\item Jeśli data\_teraz jest niewiększa niż data\_ważności wyświetl komunikat ,,Wniosek jest aktualny.'' w przeciwnym wypadku zgłaszaj błąd ,,Wniosek utracił ważność''.
	\end{enumerate}
	
\end{minipage}}
\\ \\

%\textunderbrace{Ala ma kota}{warunek}
%\textunderbrace{Text \textoverbrace{text}{aaa} text text}{bbb}

%\bigskip

%\emph{\textunderbrace{Text \textoverbrace{text}{aaa} text text}{bbb}}

%tesr\newline\\

Przyjmuję uproszczenie, że każda rozpoznawana reguła składała się będzie z trzech wyróżnialnych bloków:
\\ \\
\fbox{\begin{minipage}{40em}
\[
\underbrace{WARUNKI}  \underbrace{AKCJA\_TAK}  \underbrace{(AKCJA\_NIE)?}
\]
%S\label{fig:test}test
\end{minipage}}
\\ \\

Poszczególne bloki oddzielone będą od siebie słowami kluczowymi oznaczającymi rozpoczęcie i zakończenie bloku. \\

W celu ich wyróżnienia wprowadzam następujące oznaczenia:
\begin{enumerate}
	\item SK\_SW - Start sekcji warunku
	\item SK\_KW - Koniec sekcji warunku
	\item SK\_SAN - Start sekcji akcji wykonywanej przy niespełnionym warunku
\end{enumerate}
Schemat reguły przyjmuje następującą postać:
\\ \\
\fbox{\begin{minipage}{40em}
		\[
		\underbrace{SK\_SW} \underbrace{WARUNKI} \underbrace{SK\_KW} \underbrace{AKCJA\_TAK}  (\underbrace{SK\_SAN}\underbrace{AKCJA\_NIE)?}
		\]
		
\end{minipage}}
\\ \\

Rzut oka na przykład:
\\ \\
\fbox{\begin{minipage}{40em}
			\[
		\textunderbrace{Jeśli}{SK\_SW} 			 
		\]
		\[
		\textunderbrace{data jest mniejsza od '01-01-2019' lub data jest większa niż '01-06-2019'}{WARUNEK} 		 
		\]
		\[
		\textunderbrace{wtedy}{SK\_KW} 		
		\]
		\[ 		
		\textunderbrace{zgłoś wyjątek ,,Data spoza dopuszczonego przedziału.''}{AKCJA\_TAK}		
		\]
			\[		
		\textunderbrace{w przeciwnym wypadku }{SK\_SAN} 	
		\]
		\[			
		\textunderbrace{sprawdź regułę RS-001.}{AKCJA\_NIE} 
		\]
		
\end{minipage}}
\\ \\

\paragraph{}
Ponieważ kluczowe jest właściwe rozpoznanie sekcji warunku, chciałbym wyłączyć go przed nawias i przez chwilę skupić się wyłącznie na nim. 
\\ \\
\fbox{\begin{minipage}{40em}
		
		\[
		\textunderbrace{data jest  mniejsza od '01-01-2019' lub data jest większa niż '01-06-2019'}{WARUNEK} 		 
		\]	
		
\end{minipage}}
\\ \\

Na początek trzeba zauważyć, że powyższa sekcja składa się z dwóch niezależnych wyrażeń warunkowych połączonych operatorem logicznym \textit{lub} . Każdy z pojedynczych warunków składa się z kolei z operatora relacyjnego \textit{(jest mniejsza, jest większa)}, oraz z dwóch operandów \textit{(lewego i prawego)}.
\\
Wprowadzam więc następujące oznaczenia:
\begin{enumerate}
	\item OP\_L - Operand lewy
	\item OPR\_REL - Operator relacyjny
	\item OP\_P - Operand prawy
	\item OPR\_LOG - Operator logiczny
\end{enumerate}

Po podstawieniu, przykładowy warunek można  
\\ \\
\fbox{\begin{minipage}{40em}
		
		\[
		\textunderbrace{\textunderbrace{data}{OP\_L} \textunderbrace{jest  mniejsza od}{OPR\_REL} \textunderbrace{'01-01-2019'}{OP\_P} \textunderbrace{lub}{OPR\_LOG} \textunderbrace{data}{OP\_L} \textunderbrace{jest większa niż}{OPR\_REL} \textunderbrace{'01-06-2019'}{OP\_P}}{WARUNEK} 		 
		\]	
		
\end{minipage}}
\\ \\
Zatem zapis symboliczny sekcji warunkowej będzie wyglądał następująco:
\\ \\
\fbox{\begin{minipage}{40em}
		
		\[
		\textunderbrace{OP\_L OPR\_REL OP\_P (OPR\_LOG OP\_L OPR\_REL OP\_P)?}{WARUNEK} 		 
		\]	
		
\end{minipage}}
\\ \\

Po dokonaniu podstawienia w sekcji \textit{WARUNKI} abstrakcyjny model reguły przyjmie następującą postać:
\\ \\
\fbox{\begin{minipage}{40em}
		\[
		\underbrace{SK\_SW}
		\]
		\[	
			\textunderbrace{OP\_L OPR\_REL OP\_P (OPR\_LOG OP\_L OPR\_REL OP\_P)?}{WARUNEK}		
		\]
		\[
			\underbrace{SK\_KW}
		\]
		\[
		\underbrace{AKCJA\_TAK}
		\]
		\[
		(\underbrace{SK\_SAN}\underbrace{AKCJA\_NIE)?}
		\]
		
\end{minipage}}
\\ \\
Jeśli chodzi o akcje, to sprawa wydaje się prostsza, bo każda z nich składa się z części mówiącej o tym co ma być zrobione (AKCJA), oraz z jakim parametrem ma być wykonane (AKCJA\_PARAMETR). Ostatecznie więc, po wykonaniu podstawienia nasz model przyjmie następującą, ostateczną formę:
\\ \\
\fbox{\begin{minipage}{40em}
	\[
		\underbrace{SK\_SW}
		\]
		\[	
		\textunderbrace{OP\_L OPR\_REL OP\_P (OPR\_LOG OP\_L OPR\_REL OP\_P)?}{WARUNEK}		
		\]
		\[
		\underbrace{SK\_KW}
		\]
		\[
		\textunderbrace{AKCJA AKCJA\_PARAMETR}{AKCJA\_TAK}
		\]
		\[
		(\underbrace{SK\_SAN}\textunderbrace{AKCJA AKCJA\_PARAMETR}{AKCJA\_NIE})?
		\]
		
		\begin{enumerate}
			\item SK\_SW - Start sekcji warunku
			\item SK\_KW - Koniec sekcji warunku
			\item SK\_SAN - Start sekcji akcji wykonywanej przy niespełnionym warunku
				\item OP\_L - Operand lewy
			\item OPR\_REL - Operator relacyjny
			\item OP\_P - Operand prawy
			\item OPR\_LOG - Operator logiczny
		\end{enumerate}
		
\end{minipage}}
\\ \\
I jeszcze ostatnie spojrzenie na przykład:
\\ \\
\fbox{\begin{minipage}{40em}
		\[
		\textunderbrace{Jeśli}{SK\_SW} 			 
		\]
		\[
	\textunderbrace{\textunderbrace{data}{OP\_L} \textunderbrace{jest  mniejsza od}{OPR\_REL} \textunderbrace{'01-01-2019'}{OP\_P} \textunderbrace{lub}{OPR\_LOG} \textunderbrace{data}{OP\_L} \textunderbrace{jest większa niż}{OPR\_REL} \textunderbrace{'01-06-2019'}{OP\_P}}{WARUNEK} 		 
	\]	
		\[
		\textunderbrace{wtedy}{SK\_KW} 		
		\]
		\[ 		
		\textunderbrace{\textunderbrace{zgłoś wyjątek}{AKCJA} \textunderbrace{,,Data spoza dopuszczonego przedziału.''}{AKCJA\_PARAMETR}}{AKCJA\_TAK}		
		\]
		\[		
		\textunderbrace{w przeciwnym wypadku }{SK\_SAN} 	
		\]
		\[			
		\textunderbrace{\textunderbrace{sprawdź regułę}{AKCJA} \textunderbrace{ RS-001.}{AKCJA\_PARAMETR}}{AKCJA\_NIE} 
		\]
		
\end{minipage}}
\\ \\

Tak określony model może być podstawą do wygenerowania odpowiednio oznakowanej próbki uczącej.
