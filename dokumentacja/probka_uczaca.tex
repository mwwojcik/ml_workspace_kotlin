\section{Przygotowanie próbki uczącej}
Mechanizmy NER wchodzące w skład \textit{OpenNLP} pozwalają na stworzenie własnego modelu i przyuczenie go do rozpoznawania specyficznych bytów domenowych. Próbka ucząca jest dosyć obszernym zbiorem przykładów (dokumentacja OpenNLP mówi o minimum 15 tyś. zdań), w których w specjalny sposób otagowane zostały kluczowe frazy. 
\\ \\
\fbox{\begin{minipage}{40em}
		\small{			
		<START:SK\_SW> jeśli <END> <START:OP\_L> xxx <END> <START:OPR\_REL> jest większy niż <END> <START:OP\_P> xxx <END> <START:SK\_KW> wtedy <END> <START:AKCJA> zgłoś błąd <END> <START:AKCJA\_PARAMETR> xxx <END> .
	}			
\end{minipage}}

\paragraph{}
Byty nazwane, które ma rozpoznawać model należy umieścić pomiędzy tagami \\ \textit{<START:NAZWA\_BYTU>} i \textit{<END>}. Każda reguła moich danych uczących jest zbudowana według schematu omawianego wcześniej abstrakcyjnego modelu reguły. Zgodne z nim są również nazwy encji.
W przypadku tych części reguły, które są zmienne i specyficzne dla każdej instancji (takie jak komentarze, nazwy operandów, wszelkie parametry) użyłem frazy \textit{xxx}, która oznacza że będzie tu coś, o czym na tym etapie nie możemy nic powiedzieć (znamy tylko pozycję tego tokena względem innych encji ). 
\paragraph{}
Wygenerowanie próbki uczącej okazało się zagadnieniem samym w sobie. Do tego celu napisałem aplikację pythonową, która przetwarza zdefiniowane wektory poprawnych wartości poszczególnych encji, następnie tworzy ich iloczyny kartezjańskie i losuje do próbki zadaną ilość wygenerowanych, otagowanych reguł.

Poniżej zamieściłem wartości użyte przeze mnie do konstrukcji przykładów.
\\ \\ \\
\fbox{\begin{minipage}{40em}
	\begin{equation}
	SK\_SW=
	\begin{bmatrix}
	\text{jeśli}\\
	\text{gdy}\\
	\text{jeżeli}
	\end{bmatrix}
	\end{equation} 

\end{minipage}}
\\ \\ \\
\fbox{\begin{minipage}{40em}
	\begin{equation}
	OPR\_REL=[nie]?*
	\begin{bmatrix}
	\text{jest równy}\\
	\text{jest równa}\\
	\text{jest równe}\\
	\text{jest większy}\\
	\text{jest mniejszy}\\
	\text{jest różny}\\
	\text{jest większa}\\
	\text{jest mniejsza}\\
	\text{jest różna}\\
	\text{jest większe}\\
	\text{jest mniejsze}\\
	\text{jest różne}
	\end{bmatrix}
	\end{equation} 
\end{minipage}}
\\ \\ \\
\fbox{\begin{minipage}{40em}
	\begin{equation}
	SK\_KW=
	\begin{bmatrix}
	\text{wtedy}\\
	\text{to}
	\end{bmatrix}
	\end{equation} 
\end{minipage}}
\\ \\ \\
\fbox{\begin{minipage}{40em}
	\begin{equation}
	SK\_SAN=
	\begin{bmatrix}
	\text{w przeciwnym wypadku}
	\end{bmatrix}
	\end{equation} 
\end{minipage}}
\\ \\ \\
\fbox{\begin{minipage}{40em}
	\begin{equation}
	AKCJA=
	\begin{bmatrix}
	\text{zgłoś błąd}\\
	\text{zgłoś błąd walidacji}\\
	\text{raportuj błąd}\\
	\text{wyświetl komunikat}\\
	\text{sprawdź regułę}\\
	\text{sprawdzaj regułę}
	\end{bmatrix}
	\end{equation} 
\end{minipage}}
\\ \\ \\
Liczność próbki użytej do treningu modelu ustawiłem na poziomie ok 20 000 rekordów.