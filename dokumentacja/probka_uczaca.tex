\section{Przygotowanie próbki uczącej}
Mechanizmy NER wchodzące w skład \textit{OpenNLP} pozwalają na stworzenie własnego modelu i przyuczenie go do rozpoznawania specyficznych bytów domenowych. Próbka ucząca jest dosyć obszernym zbiorem przykładów (dokumentacja\textit{ OpenNLP} mówi o minimum 15 tyś. zdań), w których w specjalny sposób otagowane zostały kluczowe frazy. 
\\ \\
\fbox{\begin{minipage}{40em}
		\small{			
		<START:SK\_SW> jeśli <END> <START:OP\_L> xxx <END> <START:OPR\_REL> jest większy niż <END> <START:OP\_P> xxx <END> <START:SK\_KW> wtedy <END> <START:AKCJA> zgłoś błąd <END> <START:AKCJA\_PARAMETR> xxx <END> .
	}			
\end{minipage}}

\paragraph{}
Byty, które docelowo mają być rozpoznawane przez model należy umieścić pomiędzy tagami \\ \textit{<START:NAZWA\_BYTU>} i \textit{<END>}. Dodatkowo każda reguła danych uczących zbudowana jest według schematu omawianego wcześniej abstrakcyjnego modelu reguły. Zgodne z nim są również nazwy encji.
W przypadku tych części reguły, które są zmienne i specyficzne dla każdej instancji (takie jak komentarze, nazwy operandów, wszelkie parametry) użyłem frazy \textit{xxx}, która oznacza że będzie tu coś, o czym na tym etapie nie możemy nic powiedzieć (znamy tylko pozycję tego tokena względem innych encji ). 
\paragraph{}
Wygenerowanie próbki uczącej okazało się zagadnieniem samym w sobie. Do tego celu napisałem aplikację pythonową, która w danych wejściowych otrzymuje przewidywane przykładowe wartości encji,a następnie otagowuje je, tworzy ich  iloczyny kartezjańskie i ostatecznie konstruuje z nich zdanie zgodne z założonym schematem.

Poniżej znajdują się wartości poszczególnych encji użyte to wygenerowania danych uczących.

\[
\hspace*{-22em}
SK\_SW=
\begin{Bmatrix*}[l]
\text{jeśli}\\
\text{gdy}\\
\text{jeżeli}
\end{Bmatrix*}
\]

\[ 
\hspace*{-20em}
OPR\_REL=
\begin{Bmatrix*}[l]
	\text{nie}
\end{Bmatrix*}
*
\begin{Bmatrix*}[l]
	\text{jest równy}\\
	\text{jest równa}\\
	\text{jest równe}\\
	\text{jest większy}\\
	\text{jest mniejszy}\\
	\text{jest różny}\\
	\text{jest większa}\\
	\text{jest mniejsza}\\
	\text{jest różna}\\
	\text{jest większe}\\
	\text{jest mniejsze}\\
	\text{jest różne}
\end{Bmatrix*}
\begin{Bmatrix*}[l]
\text{niż}\\
\text{od}
\end{Bmatrix*}
\]

\[	
\hspace*{-22em}
SK\_KW=
	\begin{Bmatrix*}[l]
	\text{wtedy}\\
	\text{to}
\end{Bmatrix*}
\]	


\[
\hspace*{-15em}
	SK\_SAN=
	\begin{Bmatrix*}[l]
	\text{w przeciwnym wypadku}
	\end{Bmatrix*}
\]

\[
\hspace*{-16em}
\begin{array}{l}
AKCJA=
	\begin{Bmatrix*}[l]
	\text{zgłoś błąd}\\
	\text{zgłoś błąd walidacji}\\
	\text{raportuj błąd}\\
	\text{wyświetl komunikat}\\
	\text{sprawdź regułę}\\
	\text{sprawdzaj regułę}
	\end{Bmatrix*}\\

	\end{array}
\]

\paragraph{}

Przykładowe, bardziej złożone reguły utworzone opisaną wcześniej techniką:

\begin{enumerate}
	\item \small\textit{ <START:SK\_SW> jeśli <END> <START:OP\_L> xxx <END> <START:OPR\_REL> jest większy niż <END> <START:OP\_P> xxx <END> <START:SK\_KW> wtedy <END> <START:AKCJA> zgłoś błąd <END> <START:AKCJA\_PARAMETR> xxx <END>  <START:SK\_SAN> w przeciwnym wypadku <END> <START:AKCJA> sprawdzaj regułę <END> <START:AKCJA\_PARAMETR> xxx <END> .}
	\item \small\textit{ <START:SK\_SW> jeśli <END> <START:OP\_L> xxx <END> <START:OPR\_REL> nie jest mniejsza od <END> <START:OP\_P> xxx <END> <START:OPR\_LOG> oraz <END> <START:OP\_L> xxx <END> <START:OPR\_REL> nie jest równy <END> <START:OP\_P> xxx <END> <START:SK\_KW> wtedy <END> <START:AKCJA> zgłoś błąd <END> <START:AKCJA\_PARAMETR> xxx <END>  <START:SK\_SAN> w przeciwnym wypadku <END> <START:AKCJA> zgłoś błąd <END> \\ <START:AKCJA\_PARAMETR> xxx <END> .}
	
	\item \small \textit{<START:SK\_SW> jeśli <END> <START:OP\_L> xxx <END> <START:OPR\_REL> jest większy niż <END> <START:OP\_P> xxx <END> <START:OPR\_LOG> lub <END> <START:OP\_L> xxx <END> <START:OPR\_REL> jest różny od <END> <START:OP\_P> xxx <END> <START:SK\_KW> wtedy <END> <START:AKCJA> zgłoś błąd <END> <START:AKCJA\_PARAMETR> xxx <END>  <START:SK\_SAN> w przeciwnym wypadku <END> <START:AKCJA> zgłoś błąd walidacji <END> <START:AKCJA\_PARAMETR> xxx <END> .}
\end{enumerate}

Liczność próbki użytej do treningu modelu ustawiłem na poziomie ok 20 000 rekordów.